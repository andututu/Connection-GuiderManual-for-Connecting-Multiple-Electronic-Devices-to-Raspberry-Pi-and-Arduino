%! Author = andot
%! Date = 11/23/2025

% Preamble
\documentclass[11pt, a4paper]{article}

% Packages
\usepackage{amsmath}
\usepackage{ragged2e}
\usepackage{geometry}
\usepackage{graphicx}
\usepackage{hyperref}
\usepackage{array}
\usepackage{makecell}
\usepackage{float}



% Document
\title{Connection guide for Raspberry Pi and Arduino}
\author{Trieu An Do}
\begin{document}
    \maketitle\centering{BETTER FINISH IT BEFORE CHRISTMAS}

    \justifying
    \tableofcontents

    \newpage
    \newgeometry{margin = 2cm}
    \section{Motor}
        % subsection{Product link}
        % DIT CON ME MAY CHUA CO LINK
        \subsection{Connection with Raspberry Pi}
            \subsubsection{Material}
            Motor, Raspberry Pi, motor driver L298N and GPIO cables.

            L298N Pinout:

            \begin{figure}[H]                                           % 'h' = here
                \centering                                              % center the image
                \includegraphics[width=0.35\textwidth]{image/L298Npinout.jpg}
                \caption{L298N Pin Layout}                              % caption
                \label{fig: l298npin}
            \end{figure}

            Input1 controls OUT1 and so on.
            Control by H-bridge:
            \begin{center}
                \small
                \begin{tabular}{|p{2cm}|p{2cm}|p{5cm}|}
                \hline
                \vspace{-5pt}\makecell{Low} & \vspace{-5pt}\makecell{Low} & \vspace{-5pt}\makecell{Stop}\\[4pt] \hline
                \vspace{-5pt}\makecell{High} & \vspace{-5pt}\makecell{Low} & \vspace{-5pt}\makecell{Clockwise/Anti-Clockwise}\\[4pt] \hline
                \vspace{-5pt}\makecell{High} & \vspace{-5pt}\makecell{Low} & \vspace{-5pt}\makecell{Clockwise/Anti-Clockwise}\\[4pt] \hline
                \vspace{-5pt}\makecell{High} & \vspace{-5pt}\makecell{High} & \vspace{-5pt}\makecell{Brake}\\[4pt] \hline
                \end{tabular}
            \end{center}

            \subsubsection{Connection scheme}
            The connection scheme to connect motor to rbp using l298n
            \begin{figure}[H]                                           % 'h' = here
                \centering                                              % center the image
                \includegraphics[width=0.5\textwidth]{image/motorscheme.jpg}
                \caption{Connection scheme}                             % caption
                \label{fig: motorrbprscheme}
            \end{figure}

            \subsubsection{Real image}
            The image of real connection
            \begin{figure}[H]                                           % 'h' = here
                \centering                                              % center the image
                \includegraphics[width=0.5\textwidth]{image/motorrealimage.jpg}
                \caption{Real connection image}                         % caption
                \label{fig: motorrbpreal}
            \end{figure}

            \subsubsection{Remark}
            Some remark about the connection:
            \begin{itemize}
                \item 1 L298N can controll maximum 2 independent motors.
                \vspace{-5pt}\item 1 L298N occupies at least 4 GPIOs on RBP (for 2 motors).
                \vspace{-5pt}\item L298N can be powerred by RBP or external power source.

            \end{itemize}
        \subsection{Connection with Arduino}




    \section{NFC Reader(PN532 V3)}



    \section{Sensor}



    \section{Compressor}



\end{document}